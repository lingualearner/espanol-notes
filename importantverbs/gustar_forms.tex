\documentclass[grid,avery5371]{flashcards}

\usepackage[utf8]{inputenc}
\usepackage[T1]{fontenc}
\usepackage{ebgaramond}
\usepackage{verse}
\settowidth{\versewidth}{It lies behind stars and under hills,}
\addtolength{\versewidth}{2em}
\usepackage{xcolor}




\geometry{headheight=12pt,footskip=4pt}
\usepackage{fancyhdr}

\pagestyle{fancy}
\fancyhf{}
\renewcommand{\headrulewidth}{0pt}
\chead{\small Educational Flashcards of Bilbo's and Gollum's riddles, from `The Hobbit' Chapter 5 `Riddles in the Dark' by J.R.R.~Tolkien.}

\title{Riddles in the Dark}
\author{J.R.R.~Tolkien}

\cardbackstyle[\LARGE\bfseries]{plain}
\cardfrontstyle[\large]{headings}


\begin{document}
\newcommand{\frase}[1]{
    \textbf{\textcolor{red}{#1}}
}
\newcommand{\frasedos}[1]{
    \textbf{\textcolor{cyan}{#1}}
}
\newcommand{\frasetres}[1]{
    \textbf{\textcolor{teal}{#1}}
}

\newcommand{\tipo}[1]{
    \fcolorbox{red}{red!20}{#1}
}
\newcommand{\subtipo}[1]{
    \fcolorbox{yellow}{yellow!20}{\phantom{#1} }
}
\newcommand{\tipodos}[1]{
    \fcolorbox{cyan}{cyan!20}{#1}
}
\newcommand{\tipotres}[1]{
    \fcolorbox{teal}{teal!20}{#1}
}

%---------------------------------
\begin{flashcard}[
\tipodos{A mi me} \tipo{gusta} \tipotres{verbo} tercera persona
]{%
\begin{verse}[\versewidth]
\frasedos{A mi me} \frase{gusta} \frasetres{bailar} contigo \\
\end{verse}
}
.
\end{flashcard}
%---------------------------------

%---------------------------------
\begin{flashcard}[
\tipodos{ me} \tipo{importa} tercera persona
]{%
\begin{verse}[\versewidth]
\frasedos{ me } \frase{importa}  la honradez \\
No \frasedos{me} \frase{importa}\\
Me \frasedos{molestan} los restaurantes llenos\\
\end{verse}
}
.
\end{flashcard}
%---------------------------------

%---------------------------------
\begin{flashcard}[
\tipodos{ me} \tipo{importa} \tipotres{mucho} tercera persona
]{%
\begin{verse}[\versewidth]
\frasedos{me} \frase{importa} \frasetres{mucho} tu opinión.\\
No \frasedos{me} \frase{importa}\\
Me \frasedos{molestan} los restaurantes llenos\\
\end{verse}
}
.
\end{flashcard}
%---------------------------------

%---------------------------------
\begin{flashcard}[
\tipodos{ me} \tipo{importa} \tipotres{que} tercera persona
]{%
\begin{verse}[\versewidth]
\frasedos{me} \frase{importa} \frasetres{que} estés bien.\\
No \frasedos{me} \frase{importa}\\
Me \frasedos{molestan} los restaurantes llenos\\
\end{verse}
}
.
\end{flashcard}
%---------------------------------


%---------------------------------
\begin{flashcard}[
\tipodos{ te} \tipo{importa} \tipotres{si} fumo ? 
]{%
\begin{verse}[\versewidth]
\frasedos{me} \frase{importa} \frasetres{que} estés bien.\\
No \frasedos{me} \frase{importa}\\
Me \frasedos{molestan} los restaurantes llenos\\
\end{verse}
}
.
\end{flashcard}
%---------------------------------


%---------------------------------
\begin{flashcard}[
\tipodos{mi} \tipo{importa} \tipotres{lo que} frase? 
]{%
\begin{verse}[\versewidth]
No me importa lo que digan\\
\frasedos{me} \frase{importa} \frasetres{que} estés bien.\\
No \frasedos{me} \frase{importa}\\
Me \frasedos{molestan} los restaurantes llenos\\
\end{verse}
}
.
\end{flashcard}
%---------------------------------

%---------------------------------
\begin{flashcard}[
\tipodos{te} \tipo{importa} \tipotres{verb}\frasedos{me} es ? 
]{%
\begin{verse}[\versewidth]
Te importa prestarme tu libro?\\
\frasedos{me} \frase{importa} \frasetres{que} estés bien.\\
No \frasedos{me} \frase{importa}\\
Me \frasedos{molestan} los restaurantes llenos\\
\end{verse}
}
.
\end{flashcard}
%---------------------------------






\end{document}