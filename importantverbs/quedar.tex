\documentclass[grid,avery5371]{flashcards}

\usepackage[utf8]{inputenc}
\usepackage[T1]{fontenc}
\usepackage{ebgaramond}
\usepackage{verse}
\settowidth{\versewidth}{It lies behind stars and under hills,}
\addtolength{\versewidth}{2em}
\usepackage{xcolor}




\geometry{headheight=12pt,footskip=4pt}
\usepackage{fancyhdr}

\pagestyle{fancy}
\fancyhf{}
\renewcommand{\headrulewidth}{0pt}
\chead{\small Educational Flashcards of Bilbo's and Gollum's riddles, from `The Hobbit' Chapter 5 `Riddles in the Dark' by J.R.R.~Tolkien.}

\title{Riddles in the Dark}
\author{J.R.R.~Tolkien}

\cardbackstyle[\LARGE\bfseries]{plain}
\cardfrontstyle[\large]{headings}


\begin{document}
\newcommand{\frase}[1]{
    \textbf{\textcolor{red}{#1}}
}
\newcommand{\frasedos}[1]{
    \textbf{\textcolor{cyan}{#1}}
}
\newcommand{\frasetres}[1]{
    \textbf{\textcolor{teal}{#1}}
}

\newcommand{\tipo}[1]{
    \fcolorbox{red}{red!20}{#1}
}
\newcommand{\subtipo}[1]{
    \fcolorbox{yellow}{yellow!20}{\phantom{#1} }
}
\newcommand{\tipodos}[1]{
    \fcolorbox{cyan}{cyan!20}{#1}
}
\newcommand{\tipotres}[1]{
    \fcolorbox{teal}{teal!20}{#1}
}

%---------------------------------
\begin{flashcard}[
\tipodos{Gente} \tipo{ir} \tipotres{a} la \tipodos{Lugar}
]{%
\begin{verse}[\versewidth]
\frasedos{Ellos} \frase{van} \frasetres{al} \frasedos{cine} \\
\frasedos{Nosotros} \frase{vamos} \frasetres{al} \frasedos{gimnasio} \\
\frase{voy} \frasetres{a la } \frasedos{casa} de mi tia \\
\end{verse}
}
\tipodos{A} \tipotres{estar} a la \tipo{izquierda} del \tipodos{B}
\end{flashcard}
%---------------------------------


%---------------------------------
\begin{flashcard}[
\tipo{ir} \tipotres{a salir} a \tipodos{cosa/tiempo}
]{%
\begin{verse}[\versewidth]
¿\frase{vas} \frasetres{a salir} a \frasedos{cenar}? \\
\frase{voy} \frasetres{a salir } a \frasedos{trabajar} \\
¿ adonde \frase{vas} \frasetres{a salir } en la ciudad ? \\
¿ A que hora \frase{vas} \frasetres{a salir }  ? \\
\frase{voy} \frasetres{a salir} \frasedos{a las ocho} \\

\end{verse}

}
\tipodos{A} \tipotres{estar} a la  \tipo{izquierda} del \tipodos{B}
\end{flashcard}
%---------------------------------

%---------------------------------
\begin{flashcard}[
\tipo{ir} \tipotres{a ir} a la  \tipodos{tiempo/lugar/infinitivo}
]{%
\begin{verse}[\versewidth]
\frase{voy} \frasetres{a ir } a la \frasedos{playa} \\
\frase{voy} \frasetres{a ir } a \frasedos{visitar} a mi padres \\
\frase{voy} \frasetres{a ir } a \frasedos{una excursión}  \\
\end{verse}

}
\tipo{ir} \tipotres{a ir} a la  \tipodos{tiempo/lugar/infinitivo}
\end{flashcard}
%---------------------------------

%---------------------------------
\begin{flashcard}[
\tipo{voy/ir forma} \frasetres{en}  \tipodos{cosa}
]{%
\begin{verse}[\versewidth]
\frase{voy} \frasetres{en} \frasedos{tren} \\
¿\frase{vas} \frasetres{en } \frasedos{autobus} ? \\
\end{verse}

}
\tipo{voy/ir forma} \frasetres{en}  \tipodos{cosa}
\end{flashcard}
%---------------------------------

%---------------------------------
\begin{flashcard}[
\tipo{voy/ir forma} \tipotres{de}  \tipodos{A} a \tipodos{B}
]{%
\begin{verse}[\versewidth]
\frase{voy} \frasetres{de} \frasedos{España} a \frasedos{tren} \\
A mis padres se \frase{van} \frasetres{de}  \frasedos{museo} a \frasedos{su casa}\\
\frase{voy} \frasetres{de} \frasedos{la casa} al \frasedos{cuarto}
\end{verse}

}
\tipo{voy/ir forma} \frasetres{en}  \tipodos{cosa}
\end{flashcard}
%---------------------------------

%---------------------------------
\begin{flashcard}[
\tipo{ir} \tipotres{de}  \tipodos{verbo}
]{%
\begin{verse}[\versewidth]
\frase{ir} \frasetres{de} \frasedos{compras}! \\
\frase{ir} \frasetres{de} \frasedos{fiesta}! \\
\frase{ir} \frasetres{de} \frasedos{pesca}! \\
\frase{ir} \frasetres{de} \frasedos{viaje}! \\
\frase{ir} \frasetres{de} \frasedos{excursión}! \\

\end{verse}

}
\tipo{voy/ir forma} \frasetres{en}  \tipodos{cosa}
\end{flashcard}

%---------------------------------

%---------------------------------
\begin{flashcard}[
\tipodos{Gente} \tipo{se va/ van de} \tipodos{Lugar} 
]{%
\begin{verse}[\versewidth]
\frasedos{El} \frase{se va de} \frasedos{fiesta} todos los dias \\
\frasedos{Los niños} \frase{se van de} \frasedos{excurción} en verano \\

\end{verse}

}
\tipo{voy/ir forma} \frasetres{en}  \tipodos{cosa}
\end{flashcard}
%---------------------------------


\end{document}