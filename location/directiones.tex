\documentclass[grid,avery5371]{flashcards}

\usepackage[utf8]{inputenc}
\usepackage[T1]{fontenc}
\usepackage{ebgaramond}
\usepackage{verse}
\settowidth{\versewidth}{It lies behind stars and under hills,}
\addtolength{\versewidth}{2em}
\usepackage{xcolor}




\geometry{headheight=12pt,footskip=4pt}
\usepackage{fancyhdr}

\pagestyle{fancy}
\fancyhf{}
\renewcommand{\headrulewidth}{0pt}
\chead{\small Educational Flashcards of Bilbo's and Gollum's riddles, from `The Hobbit' Chapter 5 `Riddles in the Dark' by J.R.R.~Tolkien.}

\title{Riddles in the Dark}
\author{J.R.R.~Tolkien}

\cardbackstyle[\LARGE\bfseries]{plain}
\cardfrontstyle[\large]{headings}


\begin{document}
\newcommand{\frase}[1]{
    \textbf{\textcolor{red}{#1}}
}
\newcommand{\frasedos}[1]{
    \textbf{\textcolor{cyan}{#1}}
}
\newcommand{\frasetres}[1]{
    \textbf{\textcolor{teal}{#1}}
}

\newcommand{\tipo}[1]{
    \fcolorbox{red}{red!20}{#1}
}
\newcommand{\subtipo}[1]{
    \fcolorbox{yellow}{yellow!20}{\phantom{#1} }
}
\newcommand{\tipodos}[1]{
    \fcolorbox{cyan}{cyan!20}{#1}
}
\newcommand{\tipotres}[1]{
    \fcolorbox{teal}{teal!20}{#1}
}


%--------------------------------- ----
\begin{flashcard}[
\tipodos{A} \tipotres{estar} al \tipo{Northe} de la \tipodos{B}
]{%
\begin{verse}[\versewidth]
¿\frasedos{España} está al \frase{oeste} de la \frasedos{alemania}? \\
\frasedos{Centro de ciudad} está al \frase{sud} del \frasedos{museo} \\
\end{verse}
}
\tipodos{A} \tipotres{estar} al \tipo{Northe} del \tipodos{B}
\end{flashcard}
%---------------------------------

%---------------------------------
\begin{flashcard}[
\tipodos{A} \tipotres{estar} a la  \tipo{izquierda} de la \tipodos{B}
]{%
\begin{verse}[\versewidth]
¿\frasedos{Eso } está a la \frase{direcha} de la \frasedos{carratera}? \\
¿\frasedos{Él} está al \frase{izquierda} del \frasedos{museo}? \\
\end{verse}
}
\tipodos{A} \tipotres{estar} a la  \tipo{izquierda} del \tipodos{B}
\end{flashcard}
%---------------------------------


\end{document}